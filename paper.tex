\documentclass[a4paper]{article}
\usepackage{amsmath,amsthm,amssymb,mathtools}
\usepackage{enumitem}
\usepackage{changepage}
\usepackage{xcolor}
\usepackage{thmtools, thm-restate}
%\usepackage{hyperref}
\usepackage{cleveref}

% Common words containing accents and other difficult characters
\newcommand{\Holder}{H\"{o}lder}

% Semantic commands for common mathematical symbols
\newcommand{\suchthat}{\mid}
\newcommand{\into}{\hookrightarrow}
\newcommand{\from}{\colon}
\newcommand{\boundary}{\partial}
\newcommand{\union}{\cup}
\newcommand{\intersection}{\cap}
\newcommand{\bigunion}{\bigcup}
\newcommand{\bigintersection}{\bigcap}
\newcommand{\composed}{\circ}
\newcommand{\integers}{\mathbb{Z}}
\newcommand{\naturals}{\mathbb{N}}
\newcommand{\reals}{\mathbb{R}}
\newcommand{\posreals}{\mathbb{R}_{>0}}


\newcommand{\gromov}[3]{\ensuremath{\left(#2\cdot#3\right)_{#1}}}

\newcommand{\set}[1]{\left\{#1\right\}}
\newcommand{\restricted}[1]{\vert_{#1}}
\newcommand{\abs}[1]{\left|#1\right|}
\newcommand{\C}[1]{\mathrm{C}^{#1}}
\newcommand{\LC}[1]{\mathrm{LC}^{#1}}
\newcommand{\lLC}[1]{\ell\mathrm{LC}^{#1}}
\DeclarePairedDelimiter\ceil{\lceil}{\rceil}
\DeclarePairedDelimiter\floor{\lfloor}{\rfloor}
\newcommand{\isom}{\cong}
\newcommand{\disjointunion}{\amalg}

\DeclareMathOperator{\Diam}{Diam}
\DeclareMathOperator{\image}{Image}
\DeclareMathOperator{\Skel}{Sk}
\DeclareMathOperator{\sd}{sd}
\DeclareMathOperator{\dist}{d}
\DeclareMathOperator{\newdist}{\widehat{d}}
\DeclareMathOperator{\vdistance}{\rho}
\DeclareMathOperator{\Vertices}{Vert}
\DeclareMathOperator{\Ball}{Ball}
\newcommand{\homology}{\mathrm{H}}
\newcommand{\ssconverges}{\mathbin{\Rightarrow}}

% Theorem environments
\newcounter{dummy}\numberwithin{dummy}{section}

\theoremstyle{plain}
\newtheorem{lemma}[dummy]{Lemma}
\newtheorem{corollary}[dummy]{Corollary}
\newtheorem{theorem}[dummy]{Theorem}
\newtheorem{proposition}[dummy]{Proposition}
\newtheorem{conjecture}[dummy]{Conjecture}
\newtheorem{question}[dummy]{Question}
\newtheorem*{claim}{Claim}

\theoremstyle{remark}
\newtheorem{example}[dummy]{Example}
\newtheorem{remark}[dummy]{Remark}

\theoremstyle{definition}
\newtheorem{definition}[dummy]{Definition}

\def\todo#1{{\bf\color{red}{TO DO: #1}}\par}

\bibliographystyle{plain}


\title{Conditions for $d$-local-connectedness of boundaries of hyperbolic groups}
\author{Benjamin Barrett}

\begin{document}
\maketitle

\section{The Rips complex and the Gromov boundary}\label{sec:spheres}

Let $G$ be a hyperbolic group and let $S$ be a symmetric generating set for $G$.
Let $\delta \geq 0$ be such that the Cayley graph of $G$ with respect to $S$ is 
$\delta$-hyperbolic. Let $\dist(\cdot, \cdot)$ be the natural metric on the
Cayley graph.

Fix an ordering on $S$. Let $S^\star$ be the free monoid on 
$S$; order $S^\star$ by the lex-least ordering: this ordering is defined by 
requiring that shorter words precede longer and that words of equal length are 
ordered lexicographically. Let $D = 10^6\delta + 10^6$.

\begin{definition}
  We denote by $K(G)$ the \emph{Rips complex} of $G$; this is the simplicial
  complex with vertex set $G$ such that a finite set of elements of $G$ spans a
  simplex if and only if that set has diameter at most $D$ with respect to the
  word metric $d$.
\end{definition}

We will sometimes want to refer to distances between points in $K(G)$. The
precise means by which we extend the metric on $G$ to $K(G)$ will not matter
very much, so we make the following simple definition.

\begin{definition}
  For $x$ and $y$ in $K(G)$, choose vertices $x^\star$ and $y^\star$ of the
  minimal simplices containing $x$ and $y$ respectively. Then define
  $\newdist(x,y) = \dist(x^\star, y^\star)$.
\end{definition}

\begin{remark}\label{rem:dist_vs_newdist}
  The extended metric $\newdist$ is not a metric, but this will not cause a
  problem.

  Notice that $\newdist$ agrees with $\dist$ on $G$, and for any $x$ and $y$ in
  $K$, if $x^\star$ and $y^\star$ are \emph{any} vertices of simplices
  containing $x$ and $y$ respectively, then
  \begin{align*}
    \abs{\newdist(x,y) - \dist(x^\star, y^\star)} \leq 2D
  \end{align*}
\end{remark}

We denote by $\gromov{{-}}{{-}}{{-}}$ the Gromov product on $G \union
\boundary G$. Using $\newdist$ we extend the Gromov product to $K$:

\begin{definition}
  Given points $x$, $y$ and $z$ in $K$, define the \emph{Gromov product
  $\gromov{x}{y}{z}$ of $x$, $y$ and $z$} as follows.
  \begin{align*}
    \gromov{x}{y}{z} = \frac{1}{2}\left(\newdist(x,y) + \newdist(x,z) -
          \newdist(y,z)\right)
  \end{align*}
\end{definition}

We now equip $K$ with a topology as follows.

\begin{definition}
  We define a topology on $K \union \boundary G$ by describing a neighbourhood
  basis of each point.
  \begin{enumerate}
    \item For $x \in K$, the neighbourhoods of $x$ are the neighbourhoods of
      $x$ in $K$, treating $K$ with the usual topology of a simplicial complex.
    \item For $\xi \in \boundary G$, the set
      \begin{align*}
        \set{x \in K \union \boundary G \suchthat \gromov{e}{x}{\xi} \geq N}_{N\in\naturals}
      \end{align*}
      is a fundamental system of neighbourhoods for the point $x$.
  \end{enumerate}
\end{definition}

\subsection{An inverse system of complexes}

We now define a sequence of subcomplexes in $K(G)$. We will use the topology of
these complexes to approximate the topology of $\boundary G$. 

\begin{definition}
  For $n \in \naturals$ let $S_n$ be the sphere in $G$ of radius $n$, i.e.\ the 
  set $\set{g \in G \suchthat \dist(e, g) = n}$ of elements of $G$ whose 
  distance from the identity element is $n$. 

  Let $K_n$ be the full subcomplex of the Rips complex $K(G)$ with vertex set
  $S_n$.
\end{definition}

\begin{definition}
  For $n \geq m$ define a map $p^n_m \from S_n \to S_m$ as follows.
  Given an element $g$ of $S_n$ let $w_g \in S^\star$ be the lex-least 
  representative of $g$; note that this word necessarily has length $n$.
  Then define $p^n_m(g)$ to be the element of $G$ given by truncating $w_g$ to a 
  word of length $m$.
\end{definition}

\begin{lemma}\label{lem:psimplicial}
  For $n - m > D + \delta$, the map $p^n_m$ defines a simplicial map from
  $K_n \to K_m$.
\end{lemma}

\begin{proof}
  Let $\set{g_1, \dotsc, g_k}$ span a simplex in $K_n$.
  Then for $g_i$ and $g_j$ in $\set{g_1, \dotsc, g_k}$, $\dist([g_i, g_j], 
  p^n_m(g_i)) \geq n - m - D > \delta$, so $\dist(p^n_m(g_i), [e, g_j]) \leq 
  \delta$.  But $p^n_m(g_j) \in [e, g_j]$ and $\dist(e, p^n_m(g_i)) = 
  \dist(e, p^n_m(g_j))$ so $\dist(p^n_m(g_i), p^n_m(g_j)) \leq 2\delta 
  \leq D$. It follows that the diameter of $\set{p^n_m(g_1), \dotsc, 
  p^n_m(g_k)}$ is at most $D$ and therefore the set spans a simplex in $K_m$.
\end{proof}

Note that $p^m_l \composed p^n_m = p^n_l$ whenever these maps are defined, and 
therefore $(\set{K_n}_n, \{p^n_m\}_{n - m > D + \delta})$ is an inverse system 
of simplicial complexes.

\subsection{Projecting from $\boundary G$}

In order to define projections from $\boundary G$ into our system $(K_n)$, we
first describe a closely related system of simplicial complexes.

\begin{definition}
  For $x \in S_n$ let $U_n(x)$ be the set of limit points in $\boundary G$ of 
  geodesic rays $\gamma$ with $\gamma(0) = e$ and $\dist(\gamma(n), x) \leq 
  2\delta+1$.  
  
  By~\cite{bridsonhaefliger99} $U_n(x)$ is a fundamental system of 
  neighbourhoods in $\boundary G$ of the set of limit points in $\boundary G$ of 
  geodesic rays $\gamma$ with $\gamma(0) = e$ and $\gamma(n) = x$, and therefore 
  $U_n(x)$ contains an open set $V_n(x) \subset \boundary G$ containing this 
  set. Then $\set{V_n(x)}_{x \in S_n}$ is an open cover of $\boundary G$.
  
  Let $L_n$ be the nerve of this cover, so $\Vertices(L_n)$ is naturally 
  identified with a subset of $S_n$.
\end{definition}

\begin{lemma}\label{lem:L_Ksimplicial}
  The inclusion $\Skel_0 L_n \into \Skel_0 K_n^D$ is a simplicial map.
\end{lemma}

\begin{proof}
  Let $g_1, \dotsc, g_k \subset \Vertices(L_n)$ span a simplex in $L_n$, so 
  there exists a point $\xi \in \bigintersection_{i=1}^k V_n(g_i)$.  Let
  $\gamma_1, \dotsc, \gamma_n$ be geodesic rays with $\gamma_i(0) = e$,
  $\gamma_i(\infty) = \xi$ and $\dist(\gamma_i(n), g_i) \leq 2\delta+1$.
  Then $\Diam\set{\gamma_i(n) \suchthat i=1, \dotsc, n} \leq 2\delta$, and it
  follows that
  \begin{align*}
    \Diam\{g_1, \dotsc, g_k\} \leq 6\delta+ 2 \leq D.\qedhere
  \end{align*}
\end{proof}

For each $n$, fix a partition of unity subordinate to the open covering 
\begin{align*}
  \set{V_n(x) \suchthat x \in S_n}
\end{align*} 
of $\boundary G$. This is equivalent to a continuous map $\pi_n \from \boundary
G \to L_n$. Let $p^\infty_n\from\boundary G\to K_n$ be the composition of
$\pi_n$ with the simplicial map $L_n \to K_n$ given by
\cref{lem:L_Ksimplicial}.

\begin{lemma}\label{lem:close_projections}
  Let $\xi \in \boundary G$. Then $\newdist(p^\infty_n(\xi),
  p^\infty_{n+1}(\xi)) \leq 6\delta+3 + 2D$.
\end{lemma}

\begin{proof}
  Choose vertices $x_1$ and $x_2$ respectively to be vertices of minimal
  simplices of $K_n$ and $K_{n+1}$ containing $p^\infty_n(\xi)$ and
  $p^\infty_{n+1}(\xi)$. Then for $i$ equal to $1$ or $2$ there is a geodesic
  ray $\gamma_i$ with $\gamma_i(0) = e$, $\gamma_i(\infty) = \xi$, and so that
  $\dist(\gamma_1(n), x_1) \leq 2\delta+1$ and $\dist(\gamma_2(n+1),
  x_2) \leq 2\delta+1$.

  Then let $y_1 = \gamma_1(n)$ and $y_2 = \gamma_2(n+1)$. We have
  $\dist(y_1, \gamma_2) \leq 2\delta$, so by the triangle inequality
  $\dist(y_1, y_2) \leq 4\delta +1$. It follows that $\dist(x_1, x_2)
  \leq 6\delta+3$, and so the claimed inequality holds by
  \cref{rem:dist_vs_newdist}.
\end{proof}

\begin{lemma}\label{lem:boundary_gromov_product}
  Let $\alpha_1$ and $\alpha_2$ be geodesic rays with $\alpha_1(0) =
  \alpha_2(0) = e$. Then for any $m_1$ and $m_2$,
  \begin{align*}
    \gromov{e}{\alpha_1(m_1)}{\alpha_2(m_2)} \geq 
        \min\{m_1 - 3\delta, m_2 - 3\delta, \gromov{e}{\alpha_1(\infty)}{\alpha_2(\infty)} - 6\delta\}
  \end{align*}
\end{lemma}

\begin{proof}
  By~\cite{bridsonhaefliger99},
  \begin{align*}
    \gromov{e}{\alpha_1(\infty)}{\alpha_2(\infty)} \leq \liminf_{n_1, n_2}
        \gromov{e}{\alpha_1(n_1)}{\alpha_2(n_2)} + 2\delta.
  \end{align*}
  Let $n_1 \geq m_1$ and $n_2 \geq m_2$. 
  
  Let $p_1$ be a point on $[\alpha_1(n_1), \alpha_2(n_2)] \union \alpha_2$
  within a distance $\delta$ of $\alpha_1(m_1)$.  If $p_1$ is on $\alpha_2$
  then
  \begin{align*}
    \dist(\alpha_1(m_1), \alpha_2(m_2)) 
      & \leq \delta + \dist(p_1, \alpha_2(m_2)) \\
      & \leq 2\delta + \abs{m_1 - m_2}
  \end{align*}
  It follows that
  \begin{align*}
    \gromov{e}{\alpha_1(m_1)}{\alpha_2(m_2)} 
      &\geq \frac{1}{2}(m_1 + m_2 - 2\delta - \abs{m_1 - m_2})\\
      &\geq \min\{m_1, m_2\} - \delta
  \end{align*} 
  and the result follows. Otherwise $p_1$ is on $[\alpha_1(n_1),
  \alpha_2(n_2)]$. Similarly either the claimed result holds, or there is a
  point $p_2$ on $[\alpha_1(n_1), \alpha_2(n_2)]$ within a
  distance $\delta$ of $\alpha_1(m_2)$.
  
  If $p_1$ lies between $\alpha_2(n_2)$ and $p_1$ on $[\alpha_1(n_1),
  \alpha_2(n_2)]$ then, by considering the triangle with vertices $p_2$,
  $\alpha_2(m_2)$ and $\alpha_2(n_2)$, we see that $\dist(p_1, \alpha_2) \leq
  2\delta$. Therefore $\dist(\alpha_1(m_1), \alpha_2) \leq 3\delta$, and so
  \begin{align*}
    \gromov{e}{\alpha_1(m_1)}{\alpha_2(m_2)} \geq \min\{m_1, m_2\} - 3\delta
  \end{align*}
  and the result follows.

  Otherwise,
  \begin{align*}
    \dist(\alpha_1(n_1), \alpha_2(n_2)) = \dist(\alpha_1(n_1), p_1) +
        \dist(p_1, p_2) + \dist(p_2, \alpha_2(n_2)).
  \end{align*}
  Therefore,
  \begin{align*}
    \mathrlap{\gromov{e}{\alpha_1(n_1)}{\alpha_n(n_2)} - \gromov{e}{\alpha_1(m_1)}{\alpha_2(m_2)}}
    \quad\quad\phantom{{}={}}&\\
           &\mathllap{{}={}}\left(\dist(\alpha_1(m_1), \alpha_1(n_1)) - \dist(\alpha_1(m_1), p_1)\right) \\
           &+ \left(\dist(\alpha_1(m_1), \alpha_2(m_2)) - \dist(p_1, p_2)\right) \\
           &+ \left(\dist(\alpha_2(m_2), \alpha_2(n_2)) - \dist(p_2, \alpha_2(m_2))\right) \\
           &\mathllap{{}\leq{}} \delta+2\delta+\delta=4\delta 
  \end{align*}
  It follows that
  \begin{align*}
    \gromov{e}{\alpha_1(\infty)}{\alpha_2(\infty)} \leq \gromov{e}{\alpha_1(m_1)}{\alpha_2(m_2)} + 6\delta,
  \end{align*}
  which completes the proof.
\end{proof}

\begin{lemma}\label{lem:projectstoasimplex}
  Let $U$ be a subset of $\boundary G$ with diameter $\rho_0$ with respect to
  the visual metric. Let $n \leq \log_a(k_1/\rho_0)$. Then $p^\infty_n(U)$ is
  contained in a single simplex of $K_n$
\end{lemma}

\begin{proof} 
  Let $\xi_1$ and $\xi_2$ be points in $U$, so $\rho(\xi_1, \xi_2)
  \leq \rho_0$.  For $i$ equal to $1$ or $2$ let $x_i$ be a vertex of the
  minimal simplex of $K_n$ containing $p^\infty_n(\xi_i)$. Let $\alpha_i$ be a
  geodesic ray with $\alpha(0) = e$, $\alpha_i(\infty) = \xi_i$ and
  $\dist(\alpha_i(n), x_i) \leq 2\delta+1$. 
  
  By definition of the visual metric, $\gromov{e}{\xi_1}{\xi_2} \geq
  \log_a(k_1/\rho_0)$. Therefore, by \cref{lem:boundary_gromov_product},
  \begin{align*}
    \gromov{e}{\alpha_1(n)}{\alpha_2(n)} \geq \min\{n - 3\delta, \log_a(k_1/\rho_0) - 6\delta\}.
  \end{align*}
  In particular,
  \begin{align*}
    \dist(\alpha_1(n), \alpha_2(n)) \leq \max\{6\delta, 12\delta - 2(\log(k_1/\rho_0) - n)\}
  \end{align*}
  and the result follows.
\end{proof}

\begin{lemma}
  For $\xi \in \boundary G$ and $n - m \geq D + \delta$ the points $p^n_m
  \composed p^\infty_n (\xi)$ and $p^\infty_m(\xi)$ lie in a common simplex of
  $K_n^D$.
\end{lemma}

\begin{proof}
  By definition, $p^\infty_m(\xi)$ is contained in the simplex spanned by the set
  $\set{g \in S_m \suchthat \xi \in V_m(g)}$ and $p^n_m\composed p^\infty_n(\xi)$ is
  contained in the simplex spanned by $\{p^n_m(g) \suchthat g \in S_n \text{
  and } \xi \in V_n(g)\}$.  We show that the union
  \begin{align*}
    \set{g \in S_m \suchthat \xi \in V_m(g)} \union \{p^n_m(g) \suchthat g \in S_n \text{ and } \xi \in V_n(g)\}
  \end{align*}
  of these sets spans a simplex in $K_n$; this simplex then contains
  $p^\infty_m(\xi)$ and $p^n_m\composed p^\infty_n(\xi)$.  It is sufficient to
  prove that if $\xi \in V_m(g_1) \intersection V_n(g_2)$ for $g_1 \in S_m$ and
  $g_2 \in S_n$ then $\dist(g_1, p^n_m(g_2)) \leq D$.

  For $i = 1$ or $2$ let $\gamma_i$ be a geodesic ray with $\gamma_i(0) = e$ and
  $\gamma_i(\infty) = \xi$ such that $\dist(g_1, \gamma_1(m)) \leq 2\delta
  + 1$ and $\dist(g_2, \gamma_2(n)) \leq 2\delta+1$.  As in
  \cref{lem:psimplicial}, $\dist(p^n_m(g_2), \gamma_2(m)) \leq
  2\delta$, since $n-m \geq 3\delta+1$.  Then $\dist(p^n_m(g_2), g_1) \leq
  2\delta + 2\delta + 2\delta+1 \leq D$.
\end{proof}

\section{Horoballs in the Rips Complex}

In future sections we will make use of a limiting argument, which will relate
the topology of the complement of a point in the boundary $\boundary G$ to a
particular complex built from $K$, which we will think of as a horoball in $K$.
In this section we prove that this construction possesses certain properties
that we will need later.

\subsection{Defining the horoball}\label{sec:horoball}

For each $n$, let $v_n$ be a vertex of $K_n$ and let $g_n$ be an element of $G$
taken so that $g_n\cdot v_n$ is independent of $n$.  Since $K$ is locally
finite, we can pass to a subsequence so that now $v_i \in K_{n_i}$ satisfying
the following conditions:
\begin{enumerate}
  \item
    The translates by $g_i$ of large neighbourhoods of $v_i$ in the sphere
    $S_{n_i} \subset G$ are equal: for any $i \geq j$,
    \begin{equation*}
      g_i\left(N_j(v_i) \intersection S_{n_i}\right) 
            = g_j\left(N_j(v_j) \intersection S_{n_j}\right).
    \end{equation*}
    (Here $N$ denotes a neighbourhood taken with respect to the word metric on
    $G$.)
  \item
    The sequence $g_i = g_i\cdot 1$ of vertices of $K$ converges to a point
    $\xi \in \boundary G$ as $i\to\infty$.
\end{enumerate}

Let $H_i$ be the full subcomplex of $K$ with vertex set
\begin{equation*}
  g_i\left(N_i(v_i) \intersection S_{n_i}\right);
\end{equation*}
notice that $H_i$ contains $H_j$ for $i \geq j$, and also that $g_i$ maps the
full subcomplex of $K_{n_i}$ with vertex set $N_i(v_i) \intersection S_{n_i}$
isometrically onto $H_i$. 

Let $H$ be the union of the complexes $H_i \subset K$. We view $H$ as a
horosphere, the topology of which approximates that of $\boundary G - \set\xi$.
The rest of~\ref{sec:horoball} will be spent justifying this point of view.

\subsection{Geodesics and and the horosphere}

\begin{lemma}\label{lem:geodesicsmeethorosphere}
  Every bi-infinite geodesic with $\xi$ in its limit set meets $H$.
\end{lemma}

\begin{proof}
  Let $\alpha$ be a bi-infinite geodesic with $\xi$ as a limit pint. Pick a
  point $w \in f(\Vertices(J))$ and let the distance $\dist(w,\alpha)$ be
  realised by a point $x \in \alpha$.

  The points $g_i$ converge to $\xi$ as $i\to\infty$, so the distances
  $\dist(x, [\xi, g_i])$ tend to infinity as $i$ tends to infinity. The
  triangle with vertices $x$, $\xi$ and $g_i$ is $\delta$-slim, so for each $i$
  there exists a point $y_i$ on $[x, g_i]$ with $\dist(y_i, \alpha) \leq
  \delta$ and $y_i\to\xi$ as $i\to\infty$.  For each $i$ let $z_i$ be a point
  on $\alpha$ with $\dist(y_i, z_i) \delta$.

  Since $\dist(g_i, x) = n_i$ for $i$ sufficiently large and $\dist(x,y_i) \to
  \infty$ as $i\to\infty$, we have the following limit.
  \begin{align*}\label{eqn:zi_closer_to_gi_than_x}
    \dist(g_i,z_i) - n_i &\leq \dist(g_i,y_i) + \delta - n_i \text{ for $i$
                                        sufficiently large} \\
                         &= - \dist(x,y_i) + \delta \\
                         &\to -\infty\text{ as $i \to \infty$} \tag{$\star$}
  \end{align*}
  Therefore, for $i$ sufficiently large, there exists a point $p_i$ on $\alpha$
  such that $z_i$ lies between $\xi$ and $p_i$ on $\alpha$ and so that
  $\dist(g_i,p_i)=n_i$. We now show that $\dist(p_i, x)$ is uniformly bounded.

  The limit~\eqref{eqn:zi_closer_to_gi_than_x}, taken together with the fact
  that $\dist(g_i, x) = \dist(g_i, p_i) = n_i$, shows that the distances
  $\dist([z_i, g_i], x)$ and $\dist([z_i, g_i], p_i)$ tend to infinity as
  $i\to\infty$.  We now divide into cases according to which of $x$ and $p_i$
  is closer to $z_i$.
  \begin{enumerate}
    \item If $x$ is between $z_i$ and $p_i$ on $\alpha$ then by considering a
      geodesic triangle with vertices $g_i$, $z_i$ and $p_i$ we see that for
      $i$ sufficiently large, $x$ is within a distance $\delta$ of a point
      $q_i$ on the geodesic from $g_i$ to $p_i$.  Therefore,
      \begin{align*}
        \dist(g_i,x)&\leq \dist(g_i,q_i)+\delta\\
                        &\leq \dist(g_i,p_i)-\dist(p_i,q_i)+\delta\\
                        &\leq \dist(g_i,p_i)-\dist(x,p_i)+2\delta.
      \end{align*}
      It follows that $\dist(x,p_i) \leq \dist(g_i,p_i)-\dist(g_i,x)+2\delta
      = 2\delta$.
    \item If $p_i$ is between $z_i$ and $x$ on $\alpha$ then by an identical 
      argument with the r\^oles of $x$ and $p_i$ reversed we can similarly 
      deduce that 
      \begin{align*}
        \dist(x,p_i)\leq\dist(g_i,x) - \dist(g_i,p_i) + 2\delta =2\delta.
      \end{align*}
  \end{enumerate}

  Either way, $\dist(w, p_i) \leq \dist(x,w) + 2\delta$. In summary, we have
  shown that for $i$ sufficiently large there is a point $p_i\in\alpha$ with
  $\dist(g_i,p_i)=n_i$ and $\dist(p_i,f(\Vertices J))$ bounded by $\dist(x,w) +
  2\delta$. Then $p_i \in H_i$ whenever $i$ is at least $\dist(x,w) + 2\delta$.
\end{proof}

\begin{lemma}\label{lem:boundedintersection}
  Let $C$ be a compact subset of $\boundary G - \set\xi$. Then the set of points 
  of intersection between $H$ and bi-infinite geodesics with one end point equal 
  to $\xi$ and the other in $C$ is bounded.
\end{lemma}

\begin{proof}
  Let $\alpha$ be any bi-infinite geodesic with $\alpha(-\infty)=\xi$ and 
  $\alpha(\infty)\in C$. Let $x \in \alpha \intersection H$. Since $C$ is
  compact, there exists a point $v \in \alpha$, between $\xi$ and $x$ on
  $\alpha$, such for that any bi-infinite geodesic $\beta$ with
  $\beta(-\infty)$ in $C$ and $\beta(\infty) = \alpha(\infty)$, $\beta$ does
  not meet the $100\delta$-neighbourhood of the sub-ray of $\alpha$ between
  $\xi$ and $v$: otherwise a limiting argument shows that $\xi$ is in $C$.

  Since $g_i \to \xi$, the distance $\dist(v, [\xi, g_i]) \to \infty$ as
  $i\to\infty$. The triangle with vertices $\xi$, $g_i$ and $x$ is
  $\delta$-slim, so the geodesic path from $g_i$ to $x$ passes within a
  distance $\delta$ of $v$ for $i$ large enough. Therefore, since $\dist(g_i,
  x) = n_i$ for $i$ large enough, the following inequality holds for all $i$
  large enough.
  \begin{align*}
    0 \leq \dist(g_i, v) + \dist(v, x) - n_i \leq 2\delta.
  \end{align*}

  Let $\alpha'$ be a bi-infinite geodesic with $\alpha'(-\infty) = \xi$ and
  $\alpha'(\infty) \in C$ and let $x' \in \alpha' \intersection H$. The ideal
  triangle with vertices $\xi$, $\alpha(\infty)$ and $\alpha'(\infty)$ is
  $\delta$-slim, so the subray of $\alpha$ between $\xi$ and $v$ is contained
  in a $\delta$-neighbourhood of $\alpha'$.  Let $v' \in \alpha'$ be a point
  with $\dist(v, v') \leq \delta$. 
  
  We split into cases according to the order of $\xi$, $v'$ and $x'$ on
  $\alpha'$.
  \begin{enumerate}
    \item
      First we suppose that $v'$ is between $x'$ and $\xi$ on $\alpha'$. Then as
      in the argument above, for $i$ sufficiently large we have the following
      inequality:
      \begin{align*}
        0 \leq \dist(g_i, v') + \dist(v', x') - n_i \leq 2\delta.
      \end{align*}
      And therefore,
      \begin{align*}
        -2\delta \leq \dist(g_i, v) + \dist(v, x') - n_i \leq 4\delta.
      \end{align*}
      Finally, putting these inequalities together,
      \begin{align*}
        \dist(v,x') \leq \dist(v,x) + 4\delta,
      \end{align*}
      and thus we can uniformly bound this distance in this case.
    \item
      Alternatively, $x'$ is between $v'$ and $\xi$ on $\alpha'$ and so (since
      the triangle with vertices $\xi$, $v$ and $v'$ is $\delta$-slim) $x'$ is
      within a distance $2\delta$ of the geodesic ray from $x$ to $\xi$. By
      considering the triangle with vertices $\xi$, $g_i$ and $x$ for $i$ large
      enough, a geodesic segment from $g_i$ to $x$ passes within a distance
      $3\delta$ of $x'$. Therefore we have the following inequality for all $i$
      sufficiently large:
      \begin{align*}
        0 \leq \dist(g_i, x') + \dist(x', x) - \dist(g_i, x) \leq 6\delta.
      \end{align*}
      Finally, for $i$ sufficiently large, $\dist(g_i, x') = \dist(g_i, x) = n_i$,
      so
      \begin{align*}
        \dist(x,x') \leq 6\delta.
      \end{align*}
  \end{enumerate}
\end{proof}

\subsection{Projections to the horosphere}

\begin{definition}
  Let $U_i \subset \boundary G$ be the (closed) preimage of $H_i$ under the map 
  $g_i \composed p^\infty_{n_i} \composed g_i^{-1} \from \boundary G \to X$.
  Denote by $q_i$ the restriction of this map to $U_i$.
\end{definition}

The following lemma is immediate from the definition of $p^\infty_i$.

\begin{lemma}\label{lem:describing_q_i}\todo{Check this}
  Let $\zeta \in \boundary G - \set\xi$ and suppose that $q_i(\zeta) \in H_i$
  is defined, so $\zeta \in U_i$, and $N_{4\delta+1} q_i(\zeta) \in H_i$. Then
  $q_i(\zeta)$ is contained in the minimal simplex of $H_i$ with vertex set
  \begin{align*}
    N_{2\delta+1} \set{\eta(n_i)\suchthat\text{$\eta$ a geodesic ray with
    $\eta(0) = g_i$ and $\eta(\infty) = \zeta$}}.
  \end{align*}\qed
\end{lemma}

\begin{lemma}
  Every compact subset of $\boundary G - \set\xi$ is contained in $U_i$ for 
  $i$ sufficiently large.
\end{lemma} 

\begin{proof}
  Suppose that a compact subset $C$ of $\boundary G - \set\xi$ is not contained
  in $U_i$ for infinitely many $i$. Then after passing to a subsequence there
  exists for each $i$ a point $\zeta_i \in C\setminus U_i$. 

  \todo{This assumption is gone.}
  By assumption, $g_i^{-1}(H_i)$ contains the star in $K_{n_i}$ of 
  $g_i^{-1}(H_{i-1})$, so the definition of $p^\infty_{n_i}$ implies that a 
  geodesic ray $\alpha_i$ from $e$ to $g_i^{-1}(\zeta_i)$ does not pass through 
  $g_i^{-1}(H_{i-1})$. Equivalently, $g_i\composed\alpha_i$ is a geodesic from 
  $g_i$ to $\zeta_i$ that does not pass through $H_{i-1}$. 

  After reparameterising and passing to a subsequence, the geodesic rays
  $g_i\composed\alpha_i$ converge uniformly on compact subsets to a bi-infinite
  geodesic from $\xi$ to some point $\zeta \in C$.  This geodesic does not pass
  through $H$, which contradicts \cref{lem:geodesicsmeethorosphere}.
\end{proof}

\begin{lemma}\label{lem:affinehomotopic}
  For any compact subset $C\subset\boundary G - \set\xi$, there exists $i_0$ 
  such that for all $i \geq i_0$ and any $\zeta \in C$, $\newdist(q_i(\zeta),
  q_{i_0}(\zeta)) \leq 8\delta + 2$, and, in particular, $q_i\restricted{C}$
  and $q_{i_0}\restricted{C}$ are related by an affine homotopy.
\end{lemma}

\begin{proof}
  Let $B\subset H$ be a bounded set containing all points where geodesics with
  one limit point equal to $\xi$ and the other in $C$ meet $H$; this set exists
  by \cref{lem:boundedintersection}.

  Fix $\alpha$ to be some geodesic with $\alpha(-\infty)=\xi$ and
  $\alpha(\infty)\in C$ and let $x$ be a point of intersection between $\alpha$
  and $B$. Since the triangle with vertcies $x$, $g_i$ and $\xi$ is
  $\delta$-slim, and $\dist(x, [\xi, g_i]) \to \infty$ as $i\to\infty$, for
  each $i$ there exists a point $y_i$ on $[g_i, x]$ with $\dist(y_i, \alpha)
  \leq \delta$ and so that $y_i \to \xi$ as $i\to\infty$. Let $z_i \in \alpha$
  be such that $\dist(y_i, z_i)\leq\delta$.
  
  Now fix $i_0$ such that the following conditions hold for all $i\geq 
  i_0$.
  \begin{enumerate}
    \item The distance $\dist(y_i,x)$ is greater than $7\delta + \Diam B$.
    \item The set $C$ is contained in $U_i$.
  \end{enumerate}

  Let $\zeta \in C$ and let $\alpha'$ be a geodesic from $\xi$ to $\zeta$. Let 
  $x'$ be a point in the intersection of $\alpha'$ and $B$. 

  Now let $i\geq i_0$. Then by considering the ideal triangle with vertices
  $g_i$, $x$ and $x'$, and using the fact that $\dist(y_i, x) > \delta + \Diam
  B$, we deduce that there exists a point $y_i'$ on $[x', g_i]$ with
  $\dist(y_i, y_i') \leq \delta$.

  Similarly, since $\dist(z_i, x) > \delta + \Diam B$, consideration of the
  triangle with vertices $\xi$, $x$ and $x'$ reveals that there exists a point
  $z_i'$ on $\alpha'$ between $\xi$ and $x'$ with $\dist(z_i, z_i') \leq
  \delta$. Then $\dist(y_i', z_i') \leq 3\delta$.

  Since $\dist(z_i', x') > 2\delta$, the distance from $y_i'$ to the subray of
  $\alpha'$ from $x'$ to $\zeta$ is greater than $\delta$. Therefore by
  considering a triangle with vertices $g_i$, $x'$ and $\zeta$ we see that
  there exists a point $w_i$ on a geodesic ray $\eta_i$ from $g_i$ to $\zeta$
  with $\dist(w_i, y_i') \leq \delta$. It follows that $\dist(w_i, z_i') \leq
  4\delta$, and we know that $\dist(x', z_i') > 5\delta$, so $\dist(x', [z_i',
  w_i]) > \delta$ and therefore $\dist(x', \eta_i) \leq \delta$.

  In summary, any geodesic ray from $g_i$ to $\zeta$ passes within a distance 
  $\delta$ of $x'$, for $i\geq i_0$. Furthermore, $\dist(g_i,x') = n_i$, so 
  $\dist(\eta_i(n_i), x')\leq 2\delta$. It follows from the description of
  $q_i$ in \cref{lem:describing_q_i} that $q_i(\zeta)$ is contained in a
  simplex with all vertices within a distance $4\delta+1$ of $x'$. The set of
  all such vertices spans a simplex, so, in particular, $q_i\restricted{C}$ and
  $q_{i_0}\restricted{C}$ are related by an affine homotopy for $i \geq i_0$. 
\end{proof}

\begin{lemma}\label{lem:propermaps} 
  For any compact subset $C \subset H$ there exists $i_0$ such that
  \begin{align*}
    \bigunion_{i\geq i_0} q_i^{-1} C
  \end{align*}
  is a relatively compact subset of $\boundary G - \set\xi$.
\end{lemma}

\begin{proof}
  If not then there exists a sequence $\zeta_i$ of points with $\zeta_i \in U_i$ 
  for all $i$ such that $q_i(\zeta_i) \in C$ for all $i$ and $\zeta_i \to \xi$ 
  as $i \to \infty$.

  \todo{Is star actually the right word?}
  Then for each $i$ a geodesic from $g_i$ to $\zeta_i$ passes through the star 
  in $H$ of $C$.  After passing to a subsequence these geodesic rays converge to 
  a bi-infinite geodesic passing through the star in $H$ of $C$, both end points 
  of which must be $\xi$, which is a contradiction.
\end{proof}

\section{The double dagger condition}

We begin this section by recalling results of Bestvina and Mess, Bowditch, and
Swarup, which characterise the hyperbolic groups with locally (path) connected
boundary. In~\cite{bestvinamess91}, Bestvina and Mess introduce the following
condition that a hyperbolic group might satisfy.

\begin{definition}
  Double dagger condition
\end{definition}

This property is linked to the connectedness of $\boundary G$ by the following
propositions:

\begin{proposition}
  $\ddag$ implies locally path connected.
\end{proposition}

\begin{proposition}
  no cut point implies $\ddag$.
\end{proposition}

Taken together with the following theorem of Bowditch~\cite{bowditch98b} and
Swarup~\cite{swarup96}, these propositions show that $\boundary G$ is locally
path connected if and only if it is connected.

\begin{theorem}
  If $\boundary G$ is connected then $\boundary G$ does not contain a cut
  point.
\end{theorem}

\subsection{An equivalent condition}

In this section we reinterpret the double dagger condition as a condition
describing the existence of paths in the spheres $K_n$ of
Section~\ref{sec:spheres}. This interpretation gives a more systematic method
for constructing paths in $\boundary G$ from paths in the spheres $K_n$, and
permits a generalisation to higher dimensional connectivity.

\begin{definition}
  New double dagger condition. \todo{State condition.}
\end{definition}

\todo{Prove things about this condition}

\section{Local simple connectedness}

In this section we introduce a new condition that a hyperbolic group might satisfy. 

\begin{definition}
  \todo{New condition}
\end{definition}

\todo{If this condition is satisfied then the bounadry is locally simply
connected.}

\section{Simply connected point complements}

In this section we give a sufficient condition for our condition
\todo{condition} to be satisfied.

\bibliography{references}
\end{document}


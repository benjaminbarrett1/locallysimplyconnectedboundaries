\documentclass[a4paper]{article}
\usepackage{amsmath,amsthm,amssymb,mathtools}
\usepackage{enumitem}
\usepackage{changepage}
\usepackage{xcolor}
\usepackage{thmtools, thm-restate}
%\usepackage{hyperref}
\usepackage{cleveref}

% Common words containing accents and other difficult characters
\newcommand{\Holder}{H\"{o}lder}

% Semantic commands for common mathematical symbols
\newcommand{\suchthat}{\mid}
\newcommand{\into}{\hookrightarrow}
\newcommand{\from}{\colon}
\newcommand{\boundary}{\partial}
\newcommand{\union}{\cup}
\newcommand{\intersection}{\cap}
\newcommand{\bigunion}{\bigcup}
\newcommand{\bigintersection}{\bigcap}
\newcommand{\composed}{\circ}
\newcommand{\integers}{\mathbb{Z}}
\newcommand{\naturals}{\mathbb{N}}
\newcommand{\reals}{\mathbb{R}}
\newcommand{\posreals}{\mathbb{R}_{>0}}


\newcommand{\gromov}[3]{\ensuremath{\left(#2\cdot#3\right)_{#1}}}

\newcommand{\set}[1]{\left\{#1\right\}}
\newcommand{\restricted}[1]{\vert_{#1}}
\newcommand{\abs}[1]{\left|#1\right|}
\newcommand{\C}[1]{\mathrm{C}^{#1}}
\newcommand{\LC}[1]{\mathrm{LC}^{#1}}
\newcommand{\lLC}[1]{\ell\mathrm{LC}^{#1}}
\DeclarePairedDelimiter\ceil{\lceil}{\rceil}
\DeclarePairedDelimiter\floor{\lfloor}{\rfloor}
\DeclareRobustCommand{\dd}[1]{%
  \def\power{#1}
  \ifx\power\empty
    \ddag%
  \else
    \ddag^{(\power)}%
  \fi
  }
\DeclareRobustCommand{\Sd}[1]{%
  \def\power{#1}
  \ifx\power\empty
    \S%
  \else
    \S^{(\power)}%
  \fi
  }
\newcommand{\isom}{\cong}
\newcommand{\disjointunion}{\amalg}

\DeclareMathOperator{\Diam}{Diam}
\DeclareMathOperator{\image}{Image}
\DeclareMathOperator{\Skel}{Sk}
\DeclareMathOperator{\sd}{sd}
\DeclareMathOperator{\dist}{d}
\DeclareMathOperator{\newdist}{\widehat{d}}
\DeclareMathOperator{\vdistance}{\rho}
\DeclareMathOperator{\Vertices}{Vert}
\DeclareMathOperator{\Ball}{Ball}
\newcommand{\homology}{\mathrm{H}}
\newcommand{\ssconverges}{\mathbin{\Rightarrow}}

% Theorem environments
\newcounter{dummy}\numberwithin{dummy}{section}

\theoremstyle{plain}
\newtheorem{lemma}[dummy]{Lemma}
\newtheorem{corollary}[dummy]{Corollary}
\newtheorem{theorem}[dummy]{Theorem}
\newtheorem{proposition}[dummy]{Proposition}
\newtheorem{conjecture}[dummy]{Conjecture}
\newtheorem{question}[dummy]{Question}
\newtheorem*{claim}{Claim}

\theoremstyle{remark}
\newtheorem{example}[dummy]{Example}
\newtheorem{remark}[dummy]{Remark}

\theoremstyle{definition}
\newtheorem{definition}[dummy]{Definition}

\def\todo#1{{\bf\color{red}{TO DO: #1}}\par}

\bibliographystyle{plain}


\title{Conditions for $d$-local-connectedness of boundaries of hyperbolic groups}
\author{Benjamin Barrett}

\begin{document}
\maketitle

\section{The Rips complex and the Gromov boundary}\label{sec:spheres}

Let $G$ be a hyperbolic group and let $S$ be a symmetric generating set for $G$.
Let $\delta \geq 0$ be such that the Cayley graph of $G$ with respect to $S$ is 
$\delta$-hyperbolic. Let $\dist(\cdot, \cdot)$ be the natural metric on the
Cayley graph.

Fix an ordering on $S$. Let $S^\star$ be the free monoid on 
$S$; order $S^\star$ by the lex-least ordering: this ordering is defined by 
requiring that shorter words precede longer and that words of equal length are 
ordered lexicographically. Let $D = 10^6\delta + 10^6$.

\begin{definition}
  We denote by $K(G)$ the \emph{Rips complex} of $G$; this is the simplicial
  complex with vertex set $G$ such that a finite set of elements of $G$ spans a
  simplex if and only if that set has diameter at most $D$ with respect to the
  word metric $d$.
\end{definition}

We will sometimes want to refer to distances between points in $K(G)$. The
precise means by which we extend the metric on $G$ to $K(G)$ will not matter
very much, so we make the following simple definition.

\begin{definition}
  For $x$ and $y$ in $K(G)$, choose vertices $x^\star$ and $y^\star$ of the
  minimal simplices containing $x$ and $y$ respectively. Then define
  $\newdist(x,y) = \dist(x^\star, y^\star)$.
\end{definition}

\begin{remark}\label{rem:dist_vs_newdist}
  The extended metric $\newdist$ is not a metric, but this will not cause a
  problem.

  Notice that $\newdist$ agrees with $\dist$ on $G$, and for any $x$ and $y$ in
  $K$, if $x^\star$ and $y^\star$ are \emph{any} vertices of simplices
  containing $x$ and $y$ respectively, then
  \begin{align*}
    \abs{\newdist(x,y) - \dist(x^\star, y^\star)} \leq 2D
  \end{align*}
\end{remark}

We denote by $\gromov{{-}}{{-}}{{-}}$ the Gromov product on $G \union
\boundary G$. Using $\newdist$ we extend the Gromov product to $K$:

\begin{definition}
  Given points $x$, $y$ and $z$ in $K$, define the \emph{Gromov product
  $\gromov{x}{y}{z}$ of $x$, $y$ and $z$} as follows.
  \begin{align*}
    \gromov{x}{y}{z} = \frac{1}{2}\left(\newdist(x,y) + \newdist(x,z) -
          \newdist(y,z)\right)
  \end{align*}
\end{definition}

We now equip $K$ with a topology as follows.

\begin{definition}
  We define a topology on $K \union \boundary G$ by describing a neighbourhood
  basis of each point.
  \begin{enumerate}
    \item For $x \in K$, the neighbourhoods of $x$ are the neighbourhoods of
      $x$ in $K$, treating $K$ with the usual topology of a simplicial complex.
    \item For $\xi \in \boundary G$, the set
      \begin{align*}
        \set{x \in K \union \boundary G \suchthat \gromov{e}{x}{\xi} \geq N}_{N\in\naturals}
      \end{align*}
      is a fundamental system of neighbourhoods for the point $x$.
  \end{enumerate}
\end{definition}

\subsection{An inverse system of complexes}

We now define a sequence of subcomplexes in $K(G)$. We will use the topology of
these complexes to approximate the topology of $\boundary G$. 

\begin{definition}
  For $n \in \naturals$ let $S_n$ be the sphere in $G$ of radius $n$, i.e.\ the 
  set $\set{g \in G \suchthat \dist(e, g) = n}$ of elements of $G$ whose 
  distance from the identity element is $n$. 

  Let $K_n$ be the full subcomplex of the Rips complex $K(G)$ with vertex set
  $S_n$.
\end{definition}

\begin{definition}
  For $n \geq m$ define a map $p^n_m \from S_n \to S_m$ as follows.
  Given an element $g$ of $S_n$ let $w_g \in S^\star$ be the lex-least 
  representative of $g$; note that this word necessarily has length $n$.
  Then define $p^n_m(g)$ to be the element of $G$ given by truncating $w_g$ to a 
  word of length $m$.
\end{definition}

\begin{lemma}\label{lem:psimplicial}
  For $n - m > D + \delta$, the map $p^n_m$ defines a simplicial map from
  $K_n \to K_m$.
\end{lemma}

\begin{proof}
  Let $\set{g_1, \dotsc, g_k}$ span a simplex in $K_n$.
  Then for $g_i$ and $g_j$ in $\set{g_1, \dotsc, g_k}$, $\dist([g_i, g_j], 
  p^n_m(g_i)) \geq n - m - D > \delta$, so $\dist(p^n_m(g_i), [e, g_j]) \leq 
  \delta$.  But $p^n_m(g_j) \in [e, g_j]$ and $\dist(e, p^n_m(g_i)) = 
  \dist(e, p^n_m(g_j))$ so $\dist(p^n_m(g_i), p^n_m(g_j)) \leq 2\delta 
  \leq D$. It follows that the diameter of $\set{p^n_m(g_1), \dotsc, 
  p^n_m(g_k)}$ is at most $D$ and therefore the set spans a simplex in $K_m$.
\end{proof}

Note that $p^m_l \composed p^n_m = p^n_l$ whenever these maps are defined, and 
therefore $(\set{K_n}_n, \{p^n_m\}_{n - m > D + \delta})$ is an inverse system 
of simplicial complexes.

\subsection{Projecting from $\boundary G$}

In order to define projections from $\boundary G$ into our system $(K_n)$, we
first describe a closely related system of simplicial complexes.

\begin{definition}
  For $x \in S_n$ let $U_n(x)$ be the set of limit points in $\boundary G$ of 
  geodesic rays $\gamma$ with $\gamma(0) = e$ and $\dist(\gamma(n), x) \leq 
  2\delta+1$.  
  
  By~\cite{bridsonhaefliger99} $U_n(x)$ is a fundamental system of 
  neighbourhoods in $\boundary G$ of the set of limit points in $\boundary G$ of 
  geodesic rays $\gamma$ with $\gamma(0) = e$ and $\gamma(n) = x$, and therefore 
  $U_n(x)$ contains an open set $V_n(x) \subset \boundary G$ containing this 
  set. Then $\set{V_n(x)}_{x \in S_n}$ is an open cover of $\boundary G$.
  
  Let $L_n$ be the nerve of this cover, so $\Vertices(L_n)$ is naturally 
  identified with a subset of $S_n$.
\end{definition}

\begin{lemma}\label{lem:L_Ksimplicial}
  The inclusion $\Skel_0 L_n \into \Skel_0 K_n^D$ is a simplicial map.
\end{lemma}

\begin{proof}
  Let $g_1, \dotsc, g_k \subset \Vertices(L_n)$ span a simplex in $L_n$, so 
  there exists a point $\xi \in \bigintersection_{i=1}^k V_n(g_i)$.  Let
  $\gamma_1, \dotsc, \gamma_n$ be geodesic rays with $\gamma_i(0) = e$,
  $\gamma_i(\infty) = \xi$ and $\dist(\gamma_i(n), g_i) \leq 2\delta+1$.
  Then $\Diam\set{\gamma_i(n) \suchthat i=1, \dotsc, n} \leq 2\delta$, and it
  follows that
  \begin{align*}
    \Diam\{g_1, \dotsc, g_k\} \leq 6\delta+ 2 \leq D.\qedhere
  \end{align*}
\end{proof}

For each $n$, fix a partition of unity subordinate to the open covering 
\begin{align*}
  \set{V_n(x) \suchthat x \in S_n}
\end{align*} 
of $\boundary G$. This is equivalent to a continuous map $\pi_n \from \boundary
G \to L_n$. Let $p^\infty_n\from\boundary G\to K_n$ be the composition of
$\pi_n$ with the simplicial map $L_n \to K_n$ given by
\cref{lem:L_Ksimplicial}.

\begin{lemma}\label{lem:close_projections}
  Let $\xi \in \boundary G$. Then $\newdist(p^\infty_n(\xi),
  p^\infty_{n+1}(\xi)) \leq 6\delta+3 + 2D$.
\end{lemma}

\begin{proof}
  Choose vertices $x_1$ and $x_2$ respectively to be vertices of minimal
  simplices of $K_n$ and $K_{n+1}$ containing $p^\infty_n(\xi)$ and
  $p^\infty_{n+1}(\xi)$. Then for $i$ equal to $1$ or $2$ there is a geodesic
  ray $\gamma_i$ with $\gamma_i(0) = e$, $\gamma_i(\infty) = \xi$, and so that
  $\dist(\gamma_1(n), x_1) \leq 2\delta+1$ and $\dist(\gamma_2(n+1),
  x_2) \leq 2\delta+1$.

  Then let $y_1 = \gamma_1(n)$ and $y_2 = \gamma_2(n+1)$. We have
  $\dist(y_1, \gamma_2) \leq 2\delta$, so by the triangle inequality
  $\dist(y_1, y_2) \leq 4\delta +1$. It follows that $\dist(x_1, x_2)
  \leq 6\delta+3$, and so the claimed inequality holds by
  \cref{rem:dist_vs_newdist}.
\end{proof}

\begin{lemma}\label{lem:boundary_gromov_product}
  Let $\alpha_1$ and $\alpha_2$ be geodesic rays with $\alpha_1(0) =
  \alpha_2(0) = e$. Then for any $m_1$ and $m_2$,
  \begin{align*}
    \gromov{e}{\alpha_1(m_1)}{\alpha_2(m_2)} \geq 
        \min\{m_1 - 3\delta, m_2 - 3\delta, \gromov{e}{\alpha_1(\infty)}{\alpha_2(\infty)} - 6\delta\}
  \end{align*}
\end{lemma}

\begin{proof}
  By~\cite{bridsonhaefliger99},
  \begin{align*}
    \gromov{e}{\alpha_1(\infty)}{\alpha_2(\infty)} \leq \liminf_{n_1, n_2}
        \gromov{e}{\alpha_1(n_1)}{\alpha_2(n_2)} + 2\delta.
  \end{align*}
  Let $n_1 \geq m_1$ and $n_2 \geq m_2$. 
  
  Let $p_1$ be a point on $[\alpha_1(n_1), \alpha_2(n_2)] \union \alpha_2$
  within a distance $\delta$ of $\alpha_1(m_1)$.  If $p_1$ is on $\alpha_2$
  then
  \begin{align*}
    \dist(\alpha_1(m_1), \alpha_2(m_2)) 
      & \leq \delta + \dist(p_1, \alpha_2(m_2)) \\
      & \leq 2\delta + \abs{m_1 - m_2}
  \end{align*}
  It follows that
  \begin{align*}
    \gromov{e}{\alpha_1(m_1)}{\alpha_2(m_2)} 
      &\geq \frac{1}{2}(m_1 + m_2 - 2\delta - \abs{m_1 - m_2})\\
      &\geq \min\{m_1, m_2\} - \delta
  \end{align*} 
  and the result follows. Otherwise $p_1$ is on $[\alpha_1(n_1),
  \alpha_2(n_2)]$. Similarly either the claimed result holds, or there is a
  point $p_2$ on $[\alpha_1(n_1), \alpha_2(n_2)]$ within a
  distance $\delta$ of $\alpha_1(m_2)$.
  
  If $p_1$ lies between $\alpha_2(n_2)$ and $p_1$ on $[\alpha_1(n_1),
  \alpha_2(n_2)]$ then, by considering the triangle with vertices $p_2$,
  $\alpha_2(m_2)$ and $\alpha_2(n_2)$, we see that $\dist(p_1, \alpha_2) \leq
  2\delta$. Therefore $\dist(\alpha_1(m_1), \alpha_2) \leq 3\delta$, and so
  \begin{align*}
    \gromov{e}{\alpha_1(m_1)}{\alpha_2(m_2)} \geq \min\{m_1, m_2\} - 3\delta
  \end{align*}
  and the result follows.

  Otherwise,
  \begin{align*}
    \dist(\alpha_1(n_1), \alpha_2(n_2)) = \dist(\alpha_1(n_1), p_1) +
        \dist(p_1, p_2) + \dist(p_2, \alpha_2(n_2)).
  \end{align*}
  Therefore,
  \begin{align*}
    \mathrlap{\gromov{e}{\alpha_1(n_1)}{\alpha_n(n_2)} - \gromov{e}{\alpha_1(m_1)}{\alpha_2(m_2)}}
    \quad\quad\phantom{{}={}}&\\
           &\mathllap{{}={}}\left(\dist(\alpha_1(m_1), \alpha_1(n_1)) - \dist(\alpha_1(m_1), p_1)\right) \\
           &+ \left(\dist(\alpha_1(m_1), \alpha_2(m_2)) - \dist(p_1, p_2)\right) \\
           &+ \left(\dist(\alpha_2(m_2), \alpha_2(n_2)) - \dist(p_2, \alpha_2(m_2))\right) \\
           &\mathllap{{}\leq{}} \delta+2\delta+\delta=4\delta 
  \end{align*}
  It follows that
  \begin{align*}
    \gromov{e}{\alpha_1(\infty)}{\alpha_2(\infty)} \leq \gromov{e}{\alpha_1(m_1)}{\alpha_2(m_2)} + 6\delta,
  \end{align*}
  which completes the proof.
\end{proof}

\begin{lemma}\label{lem:projectstoasimplex}
  Let $U$ be a subset of $\boundary G$ with diameter $\rho_0$ with respect to
  the visual metric. Let $n \leq \log_a(k_1/\rho_0)$. Then $p^\infty_n(U)$ is
  contained in a single simplex of $K_n$
\end{lemma}

\begin{proof} 
  Let $\xi_1$ and $\xi_2$ be points in $U$, so $\rho(\xi_1, \xi_2)
  \leq \rho_0$.  For $i$ equal to $1$ or $2$ let $x_i$ be a vertex of the
  minimal simplex of $K_n$ containing $p^\infty_n(\xi_i)$. Let $\alpha_i$ be a
  geodesic ray with $\alpha(0) = e$, $\alpha_i(\infty) = \xi_i$ and
  $\dist(\alpha_i(n), x_i) \leq 2\delta+1$. 
  
  By definition of the visual metric, $\gromov{e}{\xi_1}{\xi_2} \geq
  \log_a(k_1/\rho_0)$. Therefore, by \cref{lem:boundary_gromov_product},
  \begin{align*}
    \gromov{e}{\alpha_1(n)}{\alpha_2(n)} \geq \min\{n - 3\delta, \log_a(k_1/\rho_0) - 6\delta\}.
  \end{align*}
  In particular,
  \begin{align*}
    \dist(\alpha_1(n), \alpha_2(n)) \leq \max\{6\delta, 12\delta - 2(\log(k_1/\rho_0) - n)\}
  \end{align*}
  and the result follows.
\end{proof}

\begin{lemma}
  For $\xi \in \boundary G$ and $n - m \geq D + \delta$ the points $p^n_m
  \composed p^\infty_n (\xi)$ and $p^\infty_m(\xi)$ lie in a common simplex of
  $K_n^D$.
\end{lemma}

\begin{proof}
  By definition, $p^\infty_m(\xi)$ is contained in the simplex spanned by the set
  $\set{g \in S_m \suchthat \xi \in V_m(g)}$ and $p^n_m\composed p^\infty_n(\xi)$ is
  contained in the simplex spanned by $\{p^n_m(g) \suchthat g \in S_n \text{
  and } \xi \in V_n(g)\}$.  We show that the union
  \begin{align*}
    \set{g \in S_m \suchthat \xi \in V_m(g)} \union \{p^n_m(g) \suchthat g \in S_n \text{ and } \xi \in V_n(g)\}
  \end{align*}
  of these sets spans a simplex in $K_n$; this simplex then contains
  $p^\infty_m(\xi)$ and $p^n_m\composed p^\infty_n(\xi)$.  It is sufficient to
  prove that if $\xi \in V_m(g_1) \intersection V_n(g_2)$ for $g_1 \in S_m$ and
  $g_2 \in S_n$ then $\dist(g_1, p^n_m(g_2)) \leq D$.

  For $i = 1$ or $2$ let $\gamma_i$ be a geodesic ray with $\gamma_i(0) = e$ and
  $\gamma_i(\infty) = \xi$ such that $\dist(g_1, \gamma_1(m)) \leq 2\delta
  + 1$ and $\dist(g_2, \gamma_2(n)) \leq 2\delta+1$.  As in
  \cref{lem:psimplicial}, $\dist(p^n_m(g_2), \gamma_2(m)) \leq
  2\delta$, since $n-m \geq 3\delta+1$.  Then $\dist(p^n_m(g_2), g_1) \leq
  2\delta + 2\delta + 2\delta+1 \leq D$.
\end{proof}

\section{Horoballs in the Rips Complex}

\section{The double dagger condition}

We begin this section by recalling results of Bestvina and Mess, Bowditch, and
Swarup, which characterise the hyperbolic groups with locally (path) connected
boundary. In~\cite{bestvinamess91}, Bestvina and Mess introduce the following
condition that a hyperbolic group might satisfy.

\begin{definition}
  Double dagger condition
\end{definition}

This property is linked to the connectedness of $\boundary G$ by the following
propositions:

\begin{proposition}
  $\ddag$ implies locally path connected.
\end{proposition}

\begin{proposition}
  no cut point implies $\ddag$.
\end{proposition}

Taken together with the following theorem of Bowditch~\cite{bowditch98b} and
Swarup~\cite{swarup96}, these propositions show that $\boundary G$ is locally
path connected if and only if it is connected.

\begin{theorem}
  If $\boundary G$ is connected then $\boundary G$ does not contain a cut
  point.
\end{theorem}

\subsection{An equivalent condition}

In this section we reinterpret the double dagger condition as a condition
describing the existence of paths in the spheres $K_n$ of
Section~\ref{sec:spheres}. This interpretation gives a more systematic method
for constructing paths in $\boundary G$ from paths in the spheres $K_n$, and
permits a generalisation to higher dimensional connectivity.

\begin{definition}
  New double dagger condition. \todo{State condition.}
\end{definition}

\todo{Prove things about this condition}

\section{Local simple connectedness}

In this section we introduce a new condition that a hyperbolic group might satisfy. 

\begin{definition}
  \todo{New condition}
\end{definition}

\todo{If this condition is satisfied then the bounadry is locally simply
connected.}

\section{Simply connected point complements}

In this section we give a sufficient condition for our condition
\todo{condition} to be satisfied.

\bibliography{references}
\end{document}

